\documentclass[12pt,a4j]{jarticle}

\title{\LaTeXe 入門{\thanks{\TeX{}のお勉強}} }
\author{1821000 ドレミファソラシド}
\date{\today}

\begin{document}
\maketitle

\part{LaTeX}
TeXはStanford大学のKnuth教授が製作した文書整形システムです。
ワープロが持っている表や式の記述ができるだけでなく、
印刷物としての美しさを左右するレイアウトを自動的に行なってくれます。

数学者でもあったKnuth教授が自分の著書を電子出版するために開発したもので、数式などがきれいに書け最近では出版会社でも多く用いられています。
TeXには様々なコマンドが用意されており、これを使いこなせば、ユーザの目的をいくらでも満足させる美しい印刷を行なうことができます。
しかしその一方で、コマンドの種類が多いため入門者に理解しづらい問題点もありました。

そこで、初心者でも利用しやすいようにDEC社のレスリーランポートが、
TeXのコマンドをマクロ化して、簡単なコマンドでTeXを利用できるようなシステムを設計作成しました。
これをLaTeXといいます。

\end{document}
