%\documentclass[12pt,a4j]{jarticle}
\documentclass[12pt,a4j,twocolumn]{jarticle}

%%%%%%%%%%%%%%%%%%%%
\iffalse
\topmargin      -15mm
\oddsidemargin  -15mm
\evensidemargin -15mm
\textwidth    180mm
\textheight   230mm
\fi
%%%%%%%%%%%%%%%%%%%%

\title{TeX のコマンドについて}
\author{1821000 岩谷 慶}
\date{\today}

\begin{document}

\maketitle

%この部分は表示されせん

\iffalse
複数行がまとめてコメントアウトされます。
この部分を表示させる場合は、
前後にあるコマンド(if, fi)の行を削除するか、
これらのコマンド自体をコメントアウトさせます。
\fi

\section{フォントの書体}
\subsection*{サンプル}
\hfill
{\rmfamily Roman }\par
{\sffamily Sans serif }\par
{\ttfamily Typewriter }\par
{\mcfamily 明朝体 }\par
{\gtfamily ゴシック体 }\par
{\mdseries Midiumface }\par
{\bfseries Boldface }\par
{\itshape Italic }\par
{\slshape Slanted }\par
{\scshape SMALL CAPS }\par
{\normalfont Normal Font }
\newpage
\section{フォントの大きさ}
\vfill
{\tiny テキスト}\par
{\scriptsize テキスト}\par
{\footnotesize テキスト}\par
{\small テキスト}\par
{\normalsize テキスト}\par
{\large テキスト}\par
{\Large テキスト}\par
{\LARGE テキスト}\par
{\huge テキスト}\par
{\Huge テキスト}

\section{文字の配置}

\begin{flushright}
右揃え
\end{flushright}

\begin{flushleft}
左揃え
\end{flushleft}

\begin{center}
中央揃え
\end{center}

\section{箇条書き}

\subsection{記号付箇条書き}

\begin{itemize}
\item テキスト1
\item テキスト2
\end{itemize}

\subsection{番号付箇条書き}
\begin{enumerate}
\item テキスト1
\item テキスト2
\end{enumerate}

\subsection{見出し付箇条書き}
\begin{description}
\item[メリット]弱酸性である
\item[見出し2]テキスト2
\end{description}

\section{verbatim環境}

特殊記号を本文中に表示\\

\verb+\begin{verbatim}+ テキスト \verb+\end{verbatim}+ \\

特殊記号を別行で表示\\

\begin{verbatim}

\include<stdio.h>

int main(void){

  int x, y;

  y = 2 * x;
  printf( "x = %d, y = %d\n", x, y );

  return 0;
}

\end{verbatim}

\section{数式}

\subsection{数式環境}

数式を本文中に記述 $ f(x_n)=\frac{1}{4}x_n^2 $\par
分数を大きく表示 $ f(x_n)=\displaystyle\frac{1}{4}x_n^2 $\\

\begin{eqnarray}
  ax^2 + bx + c & = & 0\\
  x_1 = \frac{-b + \sqrt{b^2 - 4ac}}{2a} \\
  x_2 = \frac{-b - \sqrt{b^2 - 4ac}}{2a}
\end{eqnarray}

数式を別行で記述(数式番号あり)
\begin{eqnarray}
y = x^2 + x + 1
\end{eqnarray}

数式を別行で記述(数式番号なし)

\[
y = x^2 + x + 1
\]

記号揃え(数式番号あり)
\begin{eqnarray}
y & = & x^2 + x   + 1 \\
z & = & x^3 + x^2 + x
\end{eqnarray}

記号揃え(数式番号なし)
\begin{eqnarray*}
y & = & x^2 + x   + 1 \\
z & = & x^3 + x^2 + x
\end{eqnarray*}

\subsection{array環境}

\[
	\begin{array}{ccc}
		a_{11} & a_{12} & a_{13} \\
		a_{21} & a_{22} & a_{23} \\
		a_{31} & a_{32} & a_{33}
	\end{array}
\]

\[
	(
	\begin{array}{ccc}
		a_{11} & a_{12} & a_{13} \\
		a_{21} & a_{22} & a_{23} \\
		a_{31} & a_{32} & a_{33}
	\end{array}
	)
\]

\[
	\left(
	\begin{array}{ccc}
		a_{11} & a_{12} & a_{13} \\
		a_{21} & a_{22} & a_{23} \\
		a_{31} & a_{32} & a_{33}
	\end{array}
	\right)
\]

\[
	\left[
	\begin{array}{ccc}
		a_{11} & a_{12} & a_{13} \\
		a_{21} & a_{22} & a_{23} \\
		a_{31} & a_{32} & a_{33}
	\end{array}
	\right]
\]

\subsection{様々な記号}
ギリシャ文字
\[
\begin{array}{cccccccc}
\alpha      & \beta     & \gamma  & \delta   & \epsilon  & \zeta   & \eta & \theta  \\
\iota       & \kappa    & \lambda & \mu      & \nu       & \xi     & o    & \pi     \\
\rho        & \sigma    & \tau    & \upsilon & \phi      & \chi    & \psi & \omega  \\
\varepsilon & \vartheta & \varpi  & \varrho  & \varsigma & \varphi &      &
\end{array}
\]

2項演算子
\[
\begin{array}{ccccccc}
\pm & \mp &\times & \div & \ast & \circ & \cdot \\
\cap & \cup & \vee & \wedge & \setminus \\
\oplus & \ominus & \otimes & \oslash & \odot \\
\end{array}
\]

関係演算子
\[
\begin{array}{cccccccc}
\le & \ll & \subset & \subseteq & \vdash & \in & \notin \\
\ge & \gg & \supset & \supseteq & \dashv & \ni \\
\equiv & \neq & \propto &
\sim & \simeq & \approx & \cong & \doteq \\
\end{array}
\]

矢印
\[
\begin{array}{cccccccc}
\leftarrow & \Leftarrow & \rightarrow & \Rightarrow & \leftrightarrow & \Leftrightarrow & \mapsto \\
\end{array}
\]

雑記号
\[
\begin{array}{cccccccc}
\Re & \Im & \partial & \angle & \infty & \forall & \exists & \neg
\end{array}
\]

関数
\[
\sin x \quad \cos x \quad \tan x
\] \[
\sinh x \quad \cosh x \quad \tanh x
\] \[
\log x \quad \ln x \quad \exp x
\] \[
\lim_{n \rightarrow \infty} a_n \quad \max_n a_n \quad \min_n a_n 
\] \[
\sum_{i=1}^{N} x_i \quad \int_{0}^{\infty} f(x) dx \quad m \bmod n
\]

\end{document}
