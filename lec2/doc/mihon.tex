\documentclass[10pt,a4j,twocolumn]{jarticle}
\usepackage{graphicx}

\title{情報リテラシー期末試験答案}
\author{1821000 東京都市太郎}
\date{平成30年5月28日}

\begin{document}
\maketitle

\section{箇条書きの問題}

\begin{description}
\item [予定] オープンキャンパス
\item [日時] 6/17
\item [場所] 世田谷キャンパス
\end{description}

\section{図の貼付の問題}

建設が終わった新6号館の外観を図\ref{fig:b6}に示す。

\begin{figure}[h]
\begin{center}
\includegraphics[width=60mm]{build6.eps}
\end{center}
\caption{新6号館}
\label{fig:b6}
\end{figure}

\section{表作成の問題}

情報科学科の各学年の在籍者を表\ref{tb:cs}に示す。

\begin{table}[h]
\caption{学科在籍者}
\label{tb:cs}
\begin{center}
\begin{tabular}{|l|c|c|c|c|}
\hline
      &  一年  & 二年 & 三年 & 四年 \\ \hline  
情科  & 132    & 133  & 100  &  89 \\
\hline
\end{tabular}
\end{center}
\end{table}

\section{数式の問題1}

$(x+1)^2=x^2+2x+1$は真である。

\section{数式の問題2}

以下に積分式を示す。
\[
 \int_{0}^{1}(x^2 + 1)dx = 1.33...
\]

\section{数式の問題3}

抵抗値$R_{1}$と$R_{2}$の2つの抵抗を並列に接続するとき、合成抵抗$R$は式(\ref{eq:r})で表される。

\begin{eqnarray}
 \frac{1}{R} = \frac{1}{R_1} + \frac{1}{R_2}
 \label{eq:r}
\end{eqnarray}

\section{参考文献の問題}

文献\cite{bibunsho}、\cite{Okumura}の著者は奥村晴彦である。

{\footnotesize
\begin{thebibliography}{9}

\bibitem{ohno}
大野義夫編,
\TeX\ 入門,
共立出版,1989.

\bibitem{nodera1}
野寺隆志,
楽々\LaTeX{},
共立出版,1990. 

\bibitem{bibunsho}
奥村晴彦,\LaTeX\ 美文書作成入門,
技術評論社,1991. 

\bibitem{Okumura}
奥村晴彦,[改訂版]\LaTeXe\ 美文書作成入門,
技術評論社,東京,2000.

\end{thebibliography}
}
\end{document}
